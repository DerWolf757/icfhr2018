
%% bare_conf_compsoc.tex
%% V1.4b
%% 2015/08/26
%% by Michael Shell
%% See:
%% http://www.michaelshell.org/
%% for current contact information.
%%
%% This is a skeleton file demonstrating the use of IEEEtran.cls
%% (requires IEEEtran.cls version 1.8b or later) with an IEEE Computer
%% Society conference paper.
%%
%% Support sites:
%% http://www.michaelshell.org/tex/ieeetran/
%% http://www.ctan.org/pkg/ieeetran
%% and
%% http://www.ieee.org/

%%*************************************************************************
%% Legal Notice:
%% This code is offered as-is without any warranty either expressed or
%% implied; without even the implied warranty of MERCHANTABILITY or
%% FITNESS FOR A PARTICULAR PURPOSE! 
%% User assumes all risk.
%% In no event shall the IEEE or any contributor to this code be liable for
%% any damages or losses, including, but not limited to, incidental,
%% consequential, or any other damages, resulting from the use or misuse
%% of any information contained here.
%%
%% All comments are the opinions of their respective authors and are not
%% necessarily endorsed by the IEEE.
%%
%% This work is distributed under the LaTeX Project Public License (LPPL)
%% ( http://www.latex-project.org/ ) version 1.3, and may be freely used,
%% distributed and modified. A copy of the LPPL, version 1.3, is included
%% in the base LaTeX documentation of all distributions of LaTeX released
%% 2003/12/01 or later.
%% Retain all contribution notices and credits.
%% ** Modified files should be clearly indicated as such, including  **
%% ** renaming them and changing author support contact information. **
%%*************************************************************************


% *** Authors should verify (and, if needed, correct) their LaTeX system  ***
% *** with the testflow diagnostic prior to trusting their LaTeX platform ***
% *** with production work. The IEEE's font choices and paper sizes can   ***
% *** trigger bugs that do not appear when using other class files.       ***                          ***
% The testflow support page is at:
% http://www.michaelshell.org/tex/testflow/



\documentclass[conference]{IEEEtran}
% Some/most Computer Society conferences require the compsoc mode option,
% but others may want the standard conference format.
%
% If IEEEtran.cls has not been installed into the LaTeX system files,
% manually specify the path to it like:
% \documentclass[conference,compsoc]{../sty/IEEEtran}





% Some very useful LaTeX packages include:
% (uncomment the ones you want to load)


% *** MISC UTILITY PACKAGES ***
%
%\usepackage{ifpdf}
% Heiko Oberdiek's ifpdf.sty is very useful if you need conditional
% compilation based on whether the output is pdf or dvi.
% usage:
% \ifpdf
%   % pdf code
% \else
%   % dvi code
% \fi
% The latest version of ifpdf.sty can be obtained from:
% http://www.ctan.org/pkg/ifpdf
% Also, note that IEEEtran.cls V1.7 and later provides a builtin
% \ifCLASSINFOpdf conditional that works the same way.
% When switching from latex to pdflatex and vice-versa, the compiler may
% have to be run twice to clear warning/error messages.

\newcommand{\RN}[1]{\uppercase\expandafter{\romannumeral#1}}

\usepackage[american]{babel}

% Hyperref links and bookmarks are not allowed by IEEE conventions
% Therefore, bookmarks are disabled below
% Links are disabled by using the NoHyper environment
% \begin{document}
% \begin{NoHyper}
%  ...
% \end{NoHyper}
% \end{document}
\usepackage[bookmarks=false]{hyperref}

\makeatletter
\@ifpackageloaded{babel}%
    {%
       \addto\extrasamerican{%
			\renewcommand*{\figureautorefname}{Fig.}%
			\renewcommand*{\tableautorefname}{Tab.}%
			\renewcommand*{\partautorefname}{Part}%
			\renewcommand*{\chapterautorefname}{Chap.}%
			\renewcommand*{\sectionautorefname}{Sec.}%
			\renewcommand*{\subsectionautorefname}{Sec.}%
			\renewcommand*{\subsubsectionautorefname}{Sec.}%     
                }%
       \addto\extrasngerman{% 
			\renewcommand*{\paragraphautorefname}{Absatz}%
			\renewcommand*{\subparagraphautorefname}{Unterabsatz}%
			\renewcommand*{\footnoteautorefname}{Fu\"snote}%
			\renewcommand*{\FancyVerbLineautorefname}{Zeile}%
			\renewcommand*{\theoremautorefname}{Theorem}%
			\renewcommand*{\appendixautorefname}{Anhang}%
			\renewcommand*{\equationautorefname}{Gleichung}%        
			\renewcommand*{\itemautorefname}{Punkt}%
                }%  
            % Fix to getting autorefs for subfigures right (thanks to Belinda Vogt for changing the definition)
            \providecommand{\subfigureautorefname}{\figureautorefname}%             
    }{\relax}
\makeatother


% *** CITATION PACKAGES ***
%
\ifCLASSOPTIONcompsoc
  % IEEE Computer Society needs nocompress option
  % requires cite.sty v4.0 or later (November 2003)
  \usepackage[nocompress]{cite}
\else
  % normal IEEE
  \usepackage{cite}
\fi
% cite.sty was written by Donald Arseneau
% V1.6 and later of IEEEtran pre-defines the format of the cite.sty package
% \cite{} output to follow that of the IEEE. Loading the cite package will
% result in citation numbers being automatically sorted and properly
% "compressed/ranged". e.g., [1], [9], [2], [7], [5], [6] without using
% cite.sty will become [1], [2], [5]--[7], [9] using cite.sty. cite.sty's
% \cite will automatically add leading space, if needed. Use cite.sty's
% noadjust option (cite.sty V3.8 and later) if you want to turn this off
% such as if a citation ever needs to be enclosed in parenthesis.
% cite.sty is already installed on most LaTeX systems. Be sure and use
% version 5.0 (2009-03-20) and later if using hyperref.sty.
% The latest version can be obtained at:
% http://www.ctan.org/pkg/cite
% The documentation is contained in the cite.sty file itself.
%
% Note that some packages require special options to format as the Computer
% Society requires. In particular, Computer Society  papers do not use
% compressed citation ranges as is done in typical IEEE papers
% (e.g., [1]-[4]). Instead, they list every citation separately in order
% (e.g., [1], [2], [3], [4]). To get the latter we need to load the cite
% package with the nocompress option which is supported by cite.sty v4.0
% and later.





% *** GRAPHICS RELATED PACKAGES ***
%
\usepackage{pgfplots}
\usepackage{tikz}
%\usepackage{subfig}
\usetikzlibrary{arrows,positioning,shapes,automata,fit,calc,decorations.pathreplacing,angles,quotes,backgrounds}
\usetikzlibrary{spy}
%\usepackage{showframe}
\definecolor{tucol1}{rgb}{1.0 0.5 0.05}
\definecolor{tucol2}{rgb}{0.52,0.72,0.10}
\definecolor{tucol3}{rgb}{0.0 0.0 0.8}
\definecolor{tucol5}{rgb}{0.8 0.8 0}
\definecolor{tucol4}{rgb}{0.8 0 0.8}
\definecolor{tucol6}{rgb}{0 0.8 0.8}

\ifCLASSINFOpdf
  % \usepackage[pdftex]{graphicx}
  % declare the path(s) where your graphic files are
  % \graphicspath{{../pdf/}{../jpeg/}}
  % and their extensions so you won't have to specify these with
  % every instance of \includegraphics
  % \DeclareGraphicsExtensions{.pdf,.jpeg,.png}
\else
  % or other class option (dvipsone, dvipdf, if not using dvips). graphicx
  % will default to the driver specified in the system graphics.cfg if no
  % driver is specified.
  % \usepackage[dvips]{graphicx}
  % declare the path(s) where your graphic files are
  % \graphicspath{{../eps/}}
  % and their extensions so you won't have to specify these with
  % every instance of \includegraphics
  % \DeclareGraphicsExtensions{.eps}
\fi
% graphicx was written by David Carlisle and Sebastian Rahtz. It is
% required if you want graphics, photos, etc. graphicx.sty is already
% installed on most LaTeX systems. The latest version and documentation
% can be obtained at: 
% http://www.ctan.org/pkg/graphicx
% Another good source of documentation is "Using Imported Graphics in
% LaTeX2e" by Keith Reckdahl which can be found at:
% http://www.ctan.org/pkg/epslatex
%
% latex, and pdflatex in dvi mode, support graphics in encapsulated
% postscript (.eps) format. pdflatex in pdf mode supports graphics
% in .pdf, .jpeg, .png and .mps (metapost) formats. Users should ensure
% that all non-photo figures use a vector format (.eps, .pdf, .mps) and
% not a bitmapped formats (.jpeg, .png). The IEEE frowns on bitmapped formats
% which can result in "jaggedy"/blurry rendering of lines and letters as
% well as large increases in file sizes.
%
% You can find documentation about the pdfTeX application at:
% http://www.tug.org/applications/pdftex





% *** MATH PACKAGES ***
%
\usepackage{amsmath}
% A popular package from the American Mathematical Society that provides
% many useful and powerful commands for dealing with mathematics.
%
% Note that the amsmath package sets \interdisplaylinepenalty to 10000
% thus preventing page breaks from occurring within multiline equations. Use:
%\interdisplaylinepenalty=2500
% after loading amsmath to restore such page breaks as IEEEtran.cls normally
% does. amsmath.sty is already installed on most LaTeX systems. The latest
% version and documentation can be obtained at:
% http://www.ctan.org/pkg/amsmath





% *** SPECIALIZED LIST PACKAGES ***
%
%\usepackage{algorithmic}
% algorithmic.sty was written by Peter Williams and Rogerio Brito.
% This package provides an algorithmic environment fo describing algorithms.
% You can use the algorithmic environment in-text or within a figure
% environment to provide for a floating algorithm. Do NOT use the algorithm
% floating environment provided by algorithm.sty (by the same authors) or
% algorithm2e.sty (by Christophe Fiorio) as the IEEE does not use dedicated
% algorithm float types and packages that provide these will not provide
% correct IEEE style captions. The latest version and documentation of
% algorithmic.sty can be obtained at:
% http://www.ctan.org/pkg/algorithms
% Also of interest may be the (relatively newer and more customizable)
% algorithmicx.sty package by Szasz Janos:
% http://www.ctan.org/pkg/algorithmicx




% *** ALIGNMENT PACKAGES ***
%
\usepackage{array}
% Frank Mittelbach's and David Carlisle's array.sty patches and improves
% the standard LaTeX2e array and tabular environments to provide better
% appearance and additional user controls. As the default LaTeX2e table
% generation code is lacking to the point of almost being broken with
% respect to the quality of the end results, all users are strongly
% advised to use an enhanced (at the very least that provided by array.sty)
% set of table tools. array.sty is already installed on most systems. The
% latest version and documentation can be obtained at:
% http://www.ctan.org/pkg/array


% IEEEtran contains the IEEEeqnarray family of commands that can be used to
% generate multiline equations as well as matrices, tables, etc., of high
% quality.




% *** SUBFIGURE PACKAGES ***
%\ifCLASSOPTIONcompsoc
%  \usepackage[caption=false,font=footnotesize,labelfont=sf,textfont=sf]{subfig}
%\else
%  \usepackage[caption=false,font=footnotesize]{subfig}
%\fi
% subfig.sty, written by Steven Douglas Cochran, is the modern replacement
% for subfigure.sty, the latter of which is no longer maintained and is
% incompatible with some LaTeX packages including fixltx2e. However,
% subfig.sty requires and automatically loads Axel Sommerfeldt's caption.sty
% which will override IEEEtran.cls' handling of captions and this will result
% in non-IEEE style figure/table captions. To prevent this problem, be sure
% and invoke subfig.sty's "caption=false" package option (available since
% subfig.sty version 1.3, 2005/06/28) as this is will preserve IEEEtran.cls
% handling of captions.
% Note that the Computer Society format requires a sans serif font rather
% than the serif font used in traditional IEEE formatting and thus the need
% to invoke different subfig.sty package options depending on whether
% compsoc mode has been enabled.
%
% The latest version and documentation of subfig.sty can be obtained at:
% http://www.ctan.org/pkg/subfig




% *** FLOAT PACKAGES ***
%
\usepackage{fixltx2e}
% fixltx2e, the successor to the earlier fix2col.sty, was written by
% Frank Mittelbach and David Carlisle. This package corrects a few problems
% in the LaTeX2e kernel, the most notable of which is that in current
% LaTeX2e releases, the ordering of single and double column floats is not
% guaranteed to be preserved. Thus, an unpatched LaTeX2e can allow a
% single column figure to be placed prior to an earlier double column
% figure.
% Be aware that LaTeX2e kernels dated 2015 and later have fixltx2e.sty's
% corrections already built into the system in which case a warning will
% be issued if an attempt is made to load fixltx2e.sty as it is no longer
% needed.
% The latest version and documentation can be found at:
% http://www.ctan.org/pkg/fixltx2e


%\usepackage{stfloats}
% stfloats.sty was written by Sigitas Tolusis. This package gives LaTeX2e
% the ability to do double column floats at the bottom of the page as well
% as the top. (e.g., "\begin{figure*}[!b]" is not normally possible in
% LaTeX2e). It also provides a command:
%\fnbelowfloat
% to enable the placement of footnotes below bottom floats (the standard
% LaTeX2e kernel puts them above bottom floats). This is an invasive package
% which rewrites many portions of the LaTeX2e float routines. It may not work
% with other packages that modify the LaTeX2e float routines. The latest
% version and documentation can be obtained at:
% http://www.ctan.org/pkg/stfloats
% Do not use the stfloats baselinefloat ability as the IEEE does not allow
% \baselineskip to stretch. Authors submitting work to the IEEE should note
% that the IEEE rarely uses double column equations and that authors should try
% to avoid such use. Do not be tempted to use the cuted.sty or midfloat.sty
% packages (also by Sigitas Tolusis) as the IEEE does not format its papers in
% such ways.
% Do not attempt to use stfloats with fixltx2e as they are incompatible.
% Instead, use Morten Hogholm'a dblfloatfix which combines the features
% of both fixltx2e and stfloats:
%
\usepackage{dblfloatfix}
% The latest version can be found at:
% http://www.ctan.org/pkg/dblfloatfix




% *** PDF, URL AND HYPERLINK PACKAGES ***
%
\usepackage{url}
% url.sty was written by Donald Arseneau. It provides better support for
% handling and breaking URLs. url.sty is already installed on most LaTeX
% systems. The latest version and documentation can be obtained at:
% http://www.ctan.org/pkg/url
% Basically, \url{my_url_here}.




% *** Do not adjust lengths that control margins, column widths, etc. ***
% *** Do not use packages that alter fonts (such as pslatex).         ***
% There should be no need to do such things with IEEEtran.cls V1.6 and later.
% (Unless specifically asked to do so by the journal or conference you plan
% to submit to, of course. )

% *** TABLE PACKAGES ***
\usepackage{booktabs}
\usepackage{tabularx}
\usepackage{multirow}
\newcolumntype{L}[1]{>{\hsize=#1\hsize\raggedright\arraybackslash}X}
\newcolumntype{R}[1]{>{\hsize=#1\hsize\raggedleft\arraybackslash}X}
\newcolumntype{C}[1]{>{\hsize=#1\hsize\centering\arraybackslash}X}

% correct bad hyphenation here
\hyphenation{op-tical net-works semi-conduc-tor}

\newcommand{\tikzhelpergrid}{
	\draw[help lines,xstep=.1,ystep=.1] (0,0) grid (1,1);
	\foreach \x in {0,1,...,9} { \node [anchor=north] at (\x/10,0) {0.\x}; }
	\foreach \y in {0,1,...,9} { \node [anchor=east] at (0,\y/10) {0.\y}; }
}

\begin{document}
% Disable links in the paper (also compare hyperref package import above)
\begin{NoHyper}
%
% paper title
% Titles are generally capitalized except for words such as a, an, and, as,
% at, but, by, for, in, nor, of, on, or, the, to and up, which are usually
% not capitalized unless they are the first or last word of the title.
% Linebreaks \\ can be used within to get better formatting as desired.
% Do not put math or special symbols in the title.
\title{Here is the Title of our Work}


% author names and affiliations
% use a multiple column layout for up to three different
% affiliations
\author{
%\IEEEauthorblockN{Author names removed for review}
%\IEEEauthorblockA{Address1\\ Address2\\
%Place\\
%Email}%
\IEEEauthorblockN{Eugen Rusakov, Sebastian Sudholt, Fabian Wolf and Gernot A. Fink}
\IEEEauthorblockA{Department of Computer Science\\ TU Dortmund University\\
44227 Dortmund, Germany\\
Email: \{firstname.lastname\}@tu-dortmund.de}%
}
%\and
%\IEEEauthorblockN{Homer Simpson}
%\IEEEauthorblockA{Twentieth Century Fox\\
%Springfield, USA\\
%Email: homer@thesimpsons.com}
%\and
%\IEEEauthorblockN{James Kirk\\ and Montgomery Scott}
%\IEEEauthorblockA{Starfleet Academy\\
%San Francisco, California 96678-2391\\
%Telephone: (800) 555--1212\\
%Fax: (888) 555--1212}}

% conference papers do not typically use \thanks and this command
% is locked out in conference mode. If really needed, such as for
% the acknowledgment of grants, issue a \IEEEoverridecommandlockouts
% after \documentclass

% for over three affiliations, or if they all won't fit within the width
% of the page (and note that there is less available width in this regard for
% compsoc conferences compared to traditional conferences), use this
% alternative format:
% 
%\author{\IEEEauthorblockN{Michael Shell\IEEEauthorrefmark{1},
%Homer Simpson\IEEEauthorrefmark{2},
%James Kirk\IEEEauthorrefmark{3}, 
%Montgomery Scott\IEEEauthorrefmark{3} and
%Eldon Tyrell\IEEEauthorrefmark{4}}
%\IEEEauthorblockA{\IEEEauthorrefmark{1}School of Electrical and Computer Engineering\\
%Georgia Institute of Technology,
%Atlanta, Georgia 30332--0250\\ Email: see http://www.michaelshell.org/contact.html}
%\IEEEauthorblockA{\IEEEauthorrefmark{2}Twentieth Century Fox, Springfield, USA\\
%Email: homer@thesimpsons.com}
%\IEEEauthorblockA{\IEEEauthorrefmark{3}Starfleet Academy, San Francisco, California 96678-2391\\
%Telephone: (800) 555--1212, Fax: (888) 555--1212}
%\IEEEauthorblockA{\IEEEauthorrefmark{4}Tyrell Inc., 123 Replicant Street, Los Angeles, California 90210--4321}}




% use for special paper notices
%\IEEEspecialpapernotice{(Invited Paper)}




% make the title area
\maketitle

% As a general rule, do not put math, special symbols or citations
% in the abstract
\begin{abstract}
	The generation of word hypotheses for segmentation-free word spotting on document level is usually subject to
	heuristic expert design. This involves strong assumptions about the visual appearance of text in the 
	document images. 
	In this paper we propose to generate hypotheses with text detectors. In order to do so, we
	present three detectors that are based on SIFT contrast scores, CNN region classification
	scores and attribute activation maps.
	The uncertainty in the detector scores is modeled with the extremal	regions method. 
	Retrieving word hypotheses is based on PHOC representations which we compute with the
	TPP-PHOCNet.
	We evaluate our method on the George Washington dataset and the ICFHR 2016 KWS competition
	benchmarks. In the evaluation we show that high word detection rates can be 
	achieved. This is a prerequisite for high retrieval performance that is competitive with the
	state-of-the-art.
\end{abstract}

% no keywords




% For peer review papers, you can put extra information on the cover
% page as needed:
% \ifCLASSOPTIONpeerreview
% \begin{center} \bfseries EDICS Category: 3-BBND \end{center}
% \fi
%
% For peerreview papers, this IEEEtran command inserts a page break and
% creates the second title. It will be ignored for other modes.
%\IEEEpeerreviewmaketitle



\section{Introduction}\label{sec:introduction}
Word spotting is an efficient method for making document images searchable. Therefore, it
provides an essential functionality for working with large document image collections. 
The approach is efficient since the search functionality is directly implemented and
not a by-product of a more complex task, typically transcription.
%In this way word spotting makes the most out of the given resources. 
%In its simplest scenario, 
Most commonly, the search query is either given as a word image in query-by-example scenarios or as 
text in query-by-string scenarios.
All word spotting methods need to either explicitly (segmentation-based) or implicitly (segmentation-free) segment the document collections into word image hypotheses.
%So-called query-by-example word
%spotting works with a single annotated sample of the query word. More annotated samples are
%required in order to cope with high variability in visual appearance and to support textual queries, i.e., query-by-string.
%The latter is typically the case for documents written by multiple authors.
State-of-the-art methods project the word images into an embedded attribute space
\cite{Almazan14} using \emph{Convolutional Neural Networks (CNN)} \cite{Sudholt17, Wilkinson17}. 
In this space, word spotting can then be accomplished through a simple nearest neighbor search.
%
For historic documents, automatic segmentation is especially challenging due to high variability in
%Word spotting is the ideal method for historic document images \cite{Llados12}. Historic
%documents are non-standardized which means that every document image collection has its very
%own characteristics w.r.t.
writing style, document layout, visual appearance of ink and paper, as well as aging artifacts. 
%Therefore, word segmentation in historic documents is especially challenging.
%In this regard historic document images pose a particular challenge.
%In order to spot words, it is required to segment the document into word hypotheses, first.
%Here, we propose different methods for indicating word occurrences.
%We explicitly model the uncertainty of our detectors in order to generate alternative word
%hypotheses.
%Word spotting methods relying on a given segmentation are called segmentation-based.
%Due to the aforementioned challenges in historic document images, it is, unfortunately,
%non-trivial to provide such a segmentation automatically.  

Segmentation methods that have been successful in modern document images, such as projection
profiles or connected components, are likely to fail for historic documents. Instead, these methods have to be
manually tuned
to the document collection's specificities. Interesting segmentation methods have been
presented in \cite{Manmatha05} and \cite{Wilkinson15}. Within the scale space approach in
\cite{Manmatha05}, some parameters can
automatically be derived from data. The approach in \cite{Wilkinson15} uses a CNN for
classifying segmentation hypotheses. The visual word appearance is, therefore, learned from
annotated sample data. 
However, methods addressing solely segmentation need to detect words without recognizing them
or, in case of word spotting, without taking relevance to the query into account.   
Therefore, these methods have to rely on discriminative characteristics of the document
collections considered. In the challenging scenario of historic document images, it
remains questionable, if suitable characteristics can automatically be extracted.
This aspect can potentially limit the generalization capability.

In order to be more robust with respect to word size variability,
our segmentation-free word spotting method is inspired by approaches using local text
detectors. In many cases text detectors are solely built on connected components, 
e.g. \cite{Rodriguez09, Kovalchuk14, Ghosh15a}. %, Riba15}. 
This has two important drawbacks. First, the
detectors are dependent on document image binarization. In historic document images
binarization is difficult due to fading ink, low contrast and inhomogeneous backgrounds. This
makes detections imprecise. Second, it can be difficult to derive word hypotheses from
connected components. Since connected components can represent parts of words, single words or
multiple words, heuristic strategies for combining connected components are required.
%Splitting connected components is hardly addressed at all, cf. \cite{Riba15} for an exception. 

For these reasons, we propose to generate word hypotheses based on higher-level feature
representations that indicate word occurrences. First, we predict scores for certain
document image regions. These scores reflect whether the respective region contains text or not. The uncertainty of these scores is then explicitely modeled with
extremal regions (ERs) \cite{Matas04} that have been very successful for text detection in
natural scene images, cf. \cite{Neumann16}. The ER approach generates hypotheses of word bounding boxes. 
For these, PHOCs are predicted using a TPP-PHOCNet \cite{Sudholt17}. This is essentially a \emph{Region-based CNN (R-CNN)} \cite{Girshick2016-RBC} framework.
After predicting the PHOCs, word spotting can be performed through a nearest neighbor search.

Generating the local text scores is a critical part of our method. Here, we consider three different approaches:
SIFT contrast scores, local region classification scores generated with a CNN and 
local word region scores obtained with an extension of CNN class activation maps \cite{Zhou2016-LDF}.
%
%In our PHOCNet-based word spotting framework 
%we consider different methods: \textbf{to be revised\dots}
%(i) SIFT contrast
%scores, (ii) local word region scores estimated with a CNN and (iii) attribute activation maps %estimated
%with a PHOCNet. The SIFT approach works without any annotated training material. Therefore, the
%scores can be interpreted as text indicators. In contrast, the
%neural network approaches need bounding boxes (ii) and bounding boxes with transcriptions
%(iii), respectively. 
%(i) local word region scores estimated with a CNN and 
%(ii) attribute activation maps (AAM) estimated with the PHOCNet. 
%In order to train the word region CNN (i), word bounding boxes are required. In contrast, the
%AAM method (ii) only requires the PHOCNet and no additional model needs to be trained.
%In all cases the uncertainty in the scores is explicitly modeled with
%extremal regions \cite{Matas04} that have been very successful for text detection in
%natural scene images, cf. \cite{Neumann16}.

%The rest of this paper is organized as follows. 
\autoref{sec:relatedwork} presents segmentation-free word spotting methods and briefly reviews extremal regions.
Our segmentation-free word
spotting approach and its evaluation are presented in \autoref{sec:method} and
\autoref{sec:evaluation}. Finally, conclusions are drawn in \autoref{sec:conclusion}.
 

%\begin{itemize}
%	\item word spotting in historic document images
%	\item segmentation-based vs segmentation-free
%	\item PHOC \cite{Almazan14}, PHOCNet \cite{Sudholt16}
%	\item hypotheses generation (patches, connected components)
%	\item contributions
%\end{itemize}

\section{Related Work}\label{sec:relatedwork}
Word spotting methods that are addressing segmentation and retrieval jointly are referred to as
segmentation-free. 
%Most prominently, hidden Markov models only require a given segmentation on
%line-level. By computing an optimal alignment of the query model and a so-called filler or
%background model, the most likely position of the query in the text line as well as the
%probability for the relevancy of the text line can be obtained, cf. \cite{Puigcerver15a}. 
In order to address the segmentation problem at document level, mainly two different approaches
can be identified. Based on local text detectors, different competing word hypotheses are obtained,
cf. \cite{Leydier09, Rodriguez09, Kovalchuk14, Ghosh15a, Wilkinson17}. 
In contrast, patch-based approaches
densely sample word hypotheses from the document images, cf. e.g., \cite{Gatos09, Almazan14a,
Rothacker15, Rusinol15, Ghosh15}. By searching the full document, patch-based approaches do not 
rely on heuristic detectors. However, they limit the search to a single patch size per
query, thus assuming that the size variability is relatively low. Finally, in both approaches word
hypotheses are ranked according to similarity with the query and overlapping hypotheses are
suppressed if they obtained a non-optimal score. Segmentation-free methods, therefore derive
the segmentation during the retrieval process and do not rely on a given segmentation that is
assumed to be correct.

Segmentation-free word spotting based on PHOC representations, cf. \cite{Almazan14}, has been presented in
\cite{Ghosh15} and \cite{Ghosh15a} for the first time.
%% hier hat Basti gekuerzt
%The document image is divided in a regular grid of blocks. Each block is
%represented with a Fisher vector. In order to obtain
%PHOC representations per block, Attribute SVM scores of the Fisher vectors are obtained for each
%PHOC component, i.e., for the different PHOC attributes. Finally, the PHOC vectors are
%projected to a low dimensional common subspace that calibrates SVMs scores and binary PHOC string
%encodings. For efficient patch-based retrieval, an
%integral image over the block-wise PHOC vectors is computed. Using the integral image is
%efficient, because a patch representation can be computed with one vector addition and two vector %subtractions. 
%The PHOC vectors can be pre-computed and representations for arbitrarily sized patch
%representation can instantly be computed at query time.
%
%In \cite{Ghosh15} a PHOC is obtained from the query word
%image by computing Attribute SVM scores for the query image's Fisher vector representation.
%After projecting the PHOC into the common subspace, the similarity between the query and the
%patches is given by the dot product between query and patch representations. 
%The most similar patches are re-ranked after adding spatial information to the Fisher vector
%representations. While the patch-based retrieval framework is applied to entire document images
%in \cite{Ghosh15}, it is only applied within document regions of interest in \cite{Ghosh15a}.
Here, the document is divided into a number of blocks and a PHOC is predicted for each block.
For efficient patch-based retrieval, an integral image over the block-wise PHOC vectors is computed.
%The effectiveness of the proposed method lies in the fact that PHOC vectors can be 
%pre-computed and representations for arbitrarily sized patch representation can instantly be computed %at query time.
In order to improve the results, a regression is learned which projects PHOCs and predictions into a common subspace. At query time, the query PHOC is projected into this subspace.
The similarity between the query and the patches is then determined through a dot product.
While all patches are considered in \cite{Ghosh15}, the approach presented in \cite{Ghosh15a} adds an indexing stage in order to efficiently detect regions of interest. In this
stage, connected components in close proximity to each other are combined in order to obtain
word hypotheses.
%Each hypothesis is indexed based on its attribute representation.
%The binarization threshold is set relative to the mean document image
%intensity. In the next step, connected components are grouped according to heuristic rules. For
%this purpose an inner connected component bounding box is defined such that it contains 90\% of the
%connected component pixels. Afterwards, connected components with intersecting inner bounding
%boxes are merged. Furthermore, horizontally close connected components are merged. Different
%thresholds are considered for creating multiple variants. In analogy to the PHOC Attribute SVM approach, 
%these word region hypotheses are represented with a bi-gram attribute vector. Based on this
%representation, each word region is indexed with an inverted file structure. 
For retrieval, candidate word
regions, obtained from the index, define the document image search area for the patch-based
framework presented in \cite{Ghosh15}. 
For query-by-example \cite{Ghosh15} the patch size equals to the size of the query word image
and for query-by-string \cite{Ghosh15a} the patch size is estimated from training word images.

Very recently, a method for proposing regions of interest and representing them with
word string embeddings in an integrated manner has been presented \cite{Wilkinson17}.
The authors train a Region Proposal Network in order to predict bounding boxes. Furthermore,
the predicted bounding boxes are augmented with a set of heuristically generated region proposals. 
A word string embedding is computed for each region. 
Regions are retrieved according to cosine distance with the query.

Related to word segmentation is text detection in natural scene images. 
These methods need to cope 
with large variability in the visual appearance of text.
While this problem domain may seem to be less constrained compared to word
segmentation, it has to be noted that the reliable detection of
word boundaries in historic document images requires to correctly recognize the text in the document
images first. In order to avoid recognition in our segmentation-free word spotting method, we
are inspired by \emph{extremal regions}. 

Extremal regions are part of the maximally stable extremal region (MSER) blob detection
method \cite{Matas04}. The key idea is to derive blobs based on connected components in 
thresholded images which are referred to as extremal regions (ER).
%. In this regard, connected components are referred to as extremal regions
%(ERs),
%since the pixel intensity values within a connected component are all smaller (or greater)
%than the pixel intensity values surrounding the connected component.% thus extremal.
%
In order to
avoid the selection of a single threshold, MSERs are detected within an ER scale space. 
This scale space is obtained by thresholding the image at all
image intensity values. 
%Across scales, ERs are organized in a tree
%structure. This is possible because ERs only change monotonically with
%monotonically changing threshold values for binarization.
%ERs are maximally stable, and therefore MSERs, if the difference of two ERs on a path in the ER tree reaches a local minimum. 

Building on the MSER approach, a method for text detection in natural scene images is presented
in \cite{Neumann16}. The method consists of different stages where character candidates are
first detected, grouped into triplets and finally merged into line regions.
%The character detection stage is based on MSERs.
For this purpose, ERs are extracted from color image channels.
In contrast to the MSER blob detection \cite{Matas04}, the ER stability is
defined on probabilistic character class scores obtained with a boosted decision tree \cite{Neumann16}. 
%rather than on the pixel-wise difference of two ER. 
The final decision whether an MSER becomes a character candidate is determined with an SVM classifier.
%The integration of character classification scores for extracting maximally stable ER, adapts the approach to the problem domain of character detection. 

In order to avoid the limitations of a basic connected component-based word detection, cf.
e.g., \cite{Kovalchuk14}, or patch-based frameworks, cf. e.g., \cite{Almazan14a}, we propose
to build ERs on top of pixel-wise text detector scores.
%However, we do not directly transfer the approach presented in \cite{Neumann16}. 
This way, we avoid the need for classifying ERs into words and non-words which would require a word recognizer.
%Obtaining is recognizer is difficult to obtain, especially given the limited ER context. 
%Instead we propose to directly build ERs on pixel-wise text detector scores. 
The main advantage over a word recognizer is that the detector is applied on the entire
document image and not limited to document image regions that have been heuristically selected. 
This way ERs model different variants for word candidates, particularly in document image
regions where the detector scores are ambiguous. In order to do so, we carefully adapt the ER
selection strategy. Furthermore, the integration and combination of different 
text detection approaches is straight forward. 

To the best of our knowledge this is the first time that
ERs are extracted based on detector scores. ERs have not been used in the context of
segmentation-free word spotting in historic document images, before.
%\begin{itemize}
%	\item segmentation-free PHOC-based word spotting with Attribute SVMs \cite{Ghosh15,
%		Ghosh15a}
%	\item maximally stable extremal regions \cite{Matas04}
%	\item Extremal region text detection \cite{Neumann16}
%	\item CNNs and class activations maps
%	\item Contributions
%\end{itemize}


%% DenseNets (Fabian)
Similar to the ResNet architecture, Dense Convolutional Networks (DenseNets) \cite{Huang2017} also utilize identity connections.
Each layer generates a new set of feature-maps, which are concatenated with the feature-maps provided by the skip-connection.
Following this connectivity scheme, each layer receives all preceeding feature-maps as an input.
The number of feature-maps added by each layer is refered to as the network's growth rate.
Already small growth rates are sufficient to achieve state-of-the-art results, resulting in very narrow layers.
To avoid the concatenation of differently sized feature-maps, DenseNet are organized into densely connected blocks.
Consequently, pooling layers are only used outside the dense blocks.
The model compactness of DenseNets is further improved by the introduction of compression layers.
A compression layer corresponds to a convolutional layer with kernel size one.
DenseNets tend to make stronger use of high-level features learned at the end of a dense block.
Therefore, the convolutional layer reduces the number of features maps by a compression factor.
The DenseNet architecture outperformed other networks such as ResNets in various state-of-the-art benchmark experiments in the field of image classification.
It has been shown, that the dense architecture is especially parameter efficient and achieves competitive results, with significantly reduced numbers of parameters.






\section{Method}\label{sec:method}


\subsection{PHOCnet Architecture}\label{sec:text_det}

 
\subsubsection{ResNet}\label{sec:text_det_sift}


\subsubsection{DenseNet}\label{sec:DenseNet}
For our experiments, we use a DenseNet with two densely connected blocks.
Before entering the first block, a convolutional layer with 32 output channels and a $2x2$ average pooling layer are applied.
Following the dense connectivity pattern, the first block consists of 30 convolutional layers with kernel sizes $3x3$ .
For the second block 60 densely connected convolutional layers are used.
The transition layer between both blocks uses a convolutional layer with kernel size $1x1$ and a $2x2$ average pooling layer.
The convolutional layer compresses the number of feature maps by a factor of $0.5$
Analogue to the \textit{TPP-PHOCNet} architecture, our DenseNet makes use of a 5-level TPP layer in combination with a Multilayer Perceptron.


\subsection{Loss Function}\label{sec:word_class}


\section{Experiments}\label{sec:evaluation}
%
For the experiments with the PHOCnet architectures we used three benchmark datasets described in \autoref{sec:datasets} 
and a evaluation protocol (\autoref{sec:protocol}) for segmentation-based wordspotting commonly used in the literature. 
In \autoref{sec:training} we describe the training setup with all network hyper-parameter used to train the networks.
Afterwards we discuss the retrieval results achieved by our methods and compare the architectures in \autoref{sec:results}.
%
%
\subsection{Datasets}\label{sec:datasets}
%
We evaluate our method on three publicly available data sets. 
The first is the \textbf{George Washington (GW) data set}. 
It consists of 20 pages that are containing 4,860 annotated words.
The pages originate from a letterbook and are quite homogeneous in their visual appearance.
However, particularly for smaller words the annotation is very sloppy. As the GW data set does
not have an official partitioning into training and test pages, we follow the common approach
and perform a four-fold cross validation. Thus, the data set is split into batches of five
consecutive documents each.

The second benchmark is the large \textbf{IAM off-line dataset} comprising 1,539 pages of 
modern handwritten English text containing 115,320 word images, written by 657 different writers. 
We used the official partition available for writer independent text line recognition. We combined the training and 
validation set to 64,XXX word images for training, and used the 13,XXX word images in the test set for evaluation. 
We exclude the stop words as queries, as this is common in this benchmark.

\textbf{Botany in British India (Botany)} is the third benchmark introduced in the Handwritting 
Keyword Spotting Competition, held during the 2016 International Conference on Frontiers in 
Handwriting Recognition. The training data of Botany was partitioned into three different 
training sets from smaller to larger (\textit{Train \RN{1}} 1684, \textit{Train \RN{2}} 5295, and \textit{Train \RN{3}} 21981 images) 
%
%
\subsection{Evaluation Protocol}\label{sec:protocol}
%
We evaluate the two CNN architectures for the data sets GW and IAM in the segmentation-based word spotting standard protocol 
proposed in \cite{Almazan14}. For the \textit{QbE} scenario all test images are considered which occure at least 
twice in the test set. For \textit{QbS} only unique string are used as queries. 
The PHOCnet predictes a attribute representation for each given query, afterwards a nearest neighbor search is performed 
by comparing the attribute vectors with the \textit{cosine similarity}. 
The Retieval list is created by sorting the computed distances from nearest to farthest. Whereas in the QbE scenario only 
predicted attribute representation are considered for a nearest neighbor search, the QbS scenario takes also the direct computed 
string embedding from the transcription (query)


\subsection{Training Setup}\label{sec:training}


\subsection{Results \& Discussion}\label{sec:results}
\begin{table*}%[]
\centering
\caption{Comparison of the different text detection methods for the Query-by-Example experiments [\%]}
\label{tab:results}
\begin{tabularx}{0.99\textwidth}{L{3.0}C{0.8}C{0.8}C{0.8}C{0.8}C{0.8}C{0.8}C{0.8}C{0.8}C{0.8}C{0.8}C{0.8}}
    \toprule
\multirow{2}{*}{Architecture}	& \multicolumn{2}{c}{George Washington} && \multicolumn{2}{c}{IAM} && \multicolumn{2}{c}{Botany?} \\ 
										& mR		& mAP					&& mR		& mAP					&& mR		& mAP					\\
\midrule
ResNet Config 1					& 73.2	& 64.8				&& 75.9	& 66.3				&& 89.9	& 86.2				\\
ResNet Config 2					& 88.4	& 80.7				&& 77.9	& 68.9				&& 91.6	& 87.1				\\
\midrule
DenseNet Config 1					& 81.8	& 77.0				&& 80.5	& 71.6				&& 81.6	& 76.1				\\
DenseNet Config 2					& 86.3	& 80.1				&& 82.2	& 73.0				&& 90.1	& 86.1				\\
\midrule
Random Attributes Config 1 ?	& 35.4	& 31.0				&& 63.6	& 53.9				&& 77.8	& 70.3				\\
Random Attributes Config 2 ?	& 67.8	& 59.9				&& 68.2	& 59.1				&& 89.5	& 83.5				\\
\bottomrule

\end{tabularx}
\end{table*}
% Experminet log files can be found in the repos experiment-logs folder.

The QbE results achieved with our word hypothesis methods are listed 
in \autoref{tab:results}. The DRs for all datasets show that we obtain very accurate
results.
High DR is a prerequisite for high retrieval performance. 
Given a query, only the hypotheses can be retrieved that have been detected, beforehand.
%Thus, DR is an upper bound for recall. Analogously, recall is an upper bound for average precision,
%as average precision measures the area under the precision-recall curve. 

An important result is that DR and retrieval performance can be improved when word
hypothesis heights are quantized to values in $[h_{min}, h_{min}+5, \cdots h_{max}]$. 
These parameters are estimated such that
$h_{max}$ is the maximum word height in the training set and $h_{min}$ is set to the typical
line height in the training set.
On the GW dataset $h_{min}$ is set to 70 pixels, to 150 pixels on Konzilsprotokolle and to 120
pixels on Botany. 
In \autoref{tab:results} these experiments are denoted with \emph{quant}. 
The positive effect has mainly three reasons. 
First, quantization is required on GW due to the sloppy annotation of smaller words
that are arbitrarily padded with white space, cf. \cite{Wilkinson17}. Accurate word hypotheses
will, therefore, not be considered as relevant. 
% Additionally such quantization has positive effect on retrieval speed. 
% This is due to the fact that the aspectratio filter prunes more of such quatized regions in average. (Great impact in experiments see Evaluation.ods!) 
Second, the TPP-PHOCNet tends to favor bounding boxes that fit the text core areas. Thus, $h_{min}$
defines a lower bound for all word hypotheses. Third, retrieval speed can be improved by
suppressing similar hypotheses.

Regarding the text detectors, we evaluate the heuristic SIFT and the learned LRC and AAM
methods.
Further, we use linear combinations of SIFT and LRC or AAM scores, as denoted with LRC+SIFT and
AAM+SIFT in \autoref{tab:results}. 
While accurate results can be achieved with SIFT, 
detection and retrieval results can be improved by adding the learning-based methods.
The best DRs are obtained with detectors including LRC. 
This is due to the explicit modeling of the visual appearance of 
word boundaries. Consequently, this mostly applies to retrieval performance as well. An
exception can be observed on GW where the training annotations for the LRC-CNN can be
considered as noisy (see above). In contrast, the AAM detector learns the visual appearance of
text. The results for the AAM detectors show that
the TPP-PHOCNet focusses on text core areas the most. Therefore, word hypothesis bounding boxes tend to fit 
closely to the words in the document.

In \autoref{tab:state-of-the-art} we consider QbE and QbS scenarios with 50\% and 25\% region overlap for 
the segmentation-free scenario. With respect to our best performing text detector configurations, the 
trend in retrieval accuracy that we observed for QbE, can also be confirmed for QbS. 
A closer look at the results for 25\% region overlap reveals that our word detections are often tighter than 
the original bounding box annotations. Word hypotheses 
that are ranked high in the retrieval list, have not been considered as relevant when using
50\% region overlap.

% Added for camera ready version.
To obtain a feasible number of word hypotheses we adjusted the number of ER-thresholds to 50 for all detectors.
In our best configuration %Regarding mAP
on GW (c.f. \autoref{tab:state-of-the-art}) around 10\,000 hypotheses per page were computed. 
After applying the aspect ratio filter approximately 5\,400 regions per query and page are left for scoring.
This low number of filtered hypotheses leads to an average query time of ~60ms per page. 
% This numbers referes to our best performing methods: GW AAM-SIFT-quant, Botany, Konzils AAM-SIFT/AAM-SIFT-quant
% SIFT @50 scales produces around 19k hypotheses on GW and Botany, 27k on Konzilsprotokolle.
% GW: AAM-SIFT-quant after aspect ratio filter -> 5,4k hypothesen -> ~89ms (Messung beinflusst durch auslastung des PCs! Alte Experimente ~60ms)  
% GW: AAM-SIFT after aspect ratio filter -> 9k hypothesen -> ~120ms
% On Botany ~300ms and Konzilsprotokolle ~280ms due to bigger phoc size.

In comparison with the state-of-the-art our results compare very favourably. We outperform the previous 
results on Botany and Konzilsprotokolle by a large margin. On GW only the very recently presented 
Region Proposal CNNs \cite{Wilkinson17} achieve better results. However, the authors use an additional CNN 
combined with brute-force hypotheses generation in order to cope with the inaccurate word annotations. 

%Their word spotting system is by orders of magnitude more complex than our straight forward R-CNN architecture.
%It has to be noted that 
%QbS performance is generally higher      
%The AAM-PHOCNet detects text most confidently in the core areas while giving not as high scores to regions featuring ascenders or decenders. Thus the bounding boxes predicted from the ER are biased towards smaller patches. While this behavior is not critical for simply indicating where certain words are, it drives down the performance of the overall system by a large margin. This is due to the fact that the overlap threshold is not met in order for a detection to count as relevant, especially for smaller words.


\begin{table*}
\centering
\caption{State of the art comparison (Results are given in mAP [\%] at different overlap thresholds)}
\label{tab:state-of-the-art}
\begin{tabularx}{.99\textwidth}{L{2.8}C{0.6}C{0.6}C{0.6}C{0.6}C{0.4}C{0.6}C{0.6}C{0.6}C{0.6}C{0.4}C{0.6}C{0.6}C{0.6}C{0.6}}
    \toprule
\multirow{3}{*}{Method}&\multicolumn{4}{c}{George Washington}&&\multicolumn{4}{c}{Botany}&&\multicolumn{4}{c}{Konzilsprotokolle}                         \\
 & \multicolumn{2}{c}{QbE} & \multicolumn{2}{c}{QbS} & & \multicolumn{2}{c}{QbE} & \multicolumn{2}{c}{QbS} && \multicolumn{2}{c}{QbE} & \multicolumn{2}{c}{QbS} \\
     		   			& 50\% & 25\% & 50\% & 25\%&   & 50\% & 25\% & 50\% & 25\% & & 50\%   & 25\%   & 50\%   & 25\%   \\
\midrule
SIFT  					& 64.8 & 71.1 & 70.7 & 76.5 &  	& 66.3 & 75.2 & 68.9 & 79.0 & 	& 86.2 & 91.1 & 84.6 & 91.5 \\
SIFT\textsubscript{quant}		& 80.7 & 90.6 & 82.5 & 89.1 &  	& 68.9 & 76.4 & 72.0 & 80.2 & 	& 87.1 & 94.0 & 87.4 & 92.9 \\
LRC+SIFT				& 78.3 & 89.3 & 81.2 & 88.9 &	& 73.0 & 79.9 & 76.2 & 83.6 & 	& 88.4 & 94.9 & 86.6 & 95.6 \\  
LRC+SIFT\textsubscript{quant} 		& 81.0 & 92.0 & 83.6 & 90.5 &  	& \textbf{74.5} & \textbf{80.4} & \textbf{78.8} & \textbf{85.3} & 	& \textbf{91.1} & 95.6 & \textbf{89.9} & 95.3 \\ 
AAM+SIFT       				& 62.3 & 69.0 & 70.0 & 76.1 &  	& 67.8 & 75.4 & 71.0 & 80.1 & 	& 84.2 & 94.9 & 81.9 & 95.2 \\ 
AAM+SIFT\textsubscript{quant} 		& 81.6 & 92.0 & 84.6 & 90.6 &  	& 69.4 & 75.9 & 74.0 & 80.3 &	& 89.6 & \textbf{96.2} & 88.9 & \textbf{96.0} \\ 
\midrule
BoF-HMM \cite{Rothacker15}    		& $-$  & $-$  & 76.5 & 80.1 &  & $-$  & $-$  & $-$  & $-$  && $-$  & $-$  & $-$  & $-$\\
Ctrl-F-Net \cite{Wilkinson17} 		& \textbf{90.9} & \textbf{97.0} & \textbf{91.0} & \textbf{95.2} &  & $-$  & $-$  & $-$  & $-$  && $-$  & $-$  & $-$  & $-$\\
TAU \cite{Pratikakis16}       		& $-$  & $-$  & $-$  & $-$  &  & 37.48& $-$  & $-$  & $-$  && 61.78& $-$  & $-$  & $-$\\
Attribute-SVMs+RR\cite{Ghosh15a}    & $-$  & $-$  & 73.7  & $-$ &  & $-$ & $-$  & $-$  & $-$ & & $-$& $-$  & $-$  & $-$\\
\bottomrule
\end{tabularx}
\end{table*}




\section{Conclusion}\label{sec:conclusion}
We have presented a method for segmentation-free word spotting which combines a novel
ER-framework with a TPP-PHOCNet in an R-CNN framework.
The ER method generates word hypotheses for which PHOCs are predicted.
We proposed three different detectors in order to predict local text scores.
This way, we avoid using a patch-based framework as well generating large amounts of region hypotheses blindly.
In the experimental evaluation we achieve results that are competitive with the state-of-the-art.





% conference papers do not normally have an appendix



% use section* for acknowledgment
%\ifCLASSOPTIONcompsoc
%  % The Computer Society usually uses the plural form
%  \section*{Acknowledgments}
%\else
%  % regular IEEE prefers the singular form
%  \section*{Acknowledgment}
%\fi
%
%
%The authors would like to thank...





% trigger a \newpage just before the given reference
% number - used to balance the columns on the last page
% adjust value as needed - may need to be readjusted if
% the document is modified later
%\IEEEtriggeratref{8}
% The "triggered" command can be changed if desired:
%\IEEEtriggercmd{\enlargethispage{-5in}}

% references section

% can use a bibliography generated by BibTeX as a .bbl file
% BibTeX documentation can be easily obtained at:
% http://mirror.ctan.org/biblio/bibtex/contrib/doc/
% The IEEEtran BibTeX style support page is at:
% http://www.michaelshell.org/tex/ieeetran/bibtex/
\bibliographystyle{IEEEtran}
% argument is your BibTeX string definitions and bibliography database(s)
%\bibliography{IEEEabrv,../bib/paper}
\bibliography{literature}
%
% <OR> manually copy in the resultant .bbl file
% set second argument of \begin to the number of references
% (used to reserve space for the reference number labels box)
%\begin{thebibliography}{1}
%
%\bibitem{IEEEhowto:kopka}
%H.~Kopka and P.~W. Daly, \emph{A Guide to \LaTeX}, 3rd~ed.\hskip 1em plus
%  0.5em minus 0.4em\relax Harlow, England: Addison-Wesley, 1999.
%
%\end{thebibliography}




\end{NoHyper}
% that's all folks
\end{document}


